\documentclass{article}
\usepackage[utf8]{inputenc}

\usepackage[T2A]{fontenc}
\usepackage[utf8]{inputenc}
\usepackage[russian]{babel}
\usepackage{geometry}

\usepackage{multienum}

\title{Русский язык и культура речи}
\author{Лисид Лаконский}
\date{February 2023}

\newtheorem{definition}{Определение}

\begin{document}
\raggedright

\maketitle
\tableofcontents
\pagebreak

\section{Русский язык и культура речи — 02.02.2023}

Преподаватель — \textbf{Горшкова Дарья Ивановна}.

На практических занятиях будут проводиться \textbf{проверочные работы} — накопительный зачёт.

\subsection{История русского языка}

История языка — \textbf{отражение истории народа}, который говорит на данном языке.

В русском языке много заимствований — отражение, с какими народами мы контактировали в разные временные периоды. Есть также \textbf{обратное влияние языка на народ}.

Языковое \textbf{родство между славянскими народами — одна языковая семья}. Славянские языки входят в \textbf{индоевропейскую языковую семью}.

\textbf{Теория происхождения славянских языков}: раньше были единым племенем, говорили на индоевропейском — постепенная миграция, распад племён — выделилсь предки современных языков.

\textbf{Три ветви} славянских языков: \textbf{восточнославянские языки, западнославянские языки, южнославянские языки}.

Образование славянской народности относят к третьему тысячелетию до нашей эры — \textbf{«праславянский язык»}.

\hfill

На базе языковой общности восточных славян возникла \textbf{Киевская Русь} — общий язык, из которых потом выделились русский, укранский, белорусский — бесписьменный язык.

Появление письменности в Киевской Руси связано с \textbf{крещением} — византийское влияние — \textbf{глаголица} — основанная на фонетическом ряде.

\textbf{Кириллица} появилась позже, после смерти Константина и Мефодия — создали его ученики.

После принятия христианства в \textbf{Киевскую Русь} начали поступать религиозные книги, написанные кириллицей.

Существовало две разновидности языка: \textbf{древнерусский язык}, используемый в быту; и язык, используемый в религиозном суждении, книжный — \textbf{старославянский}, \textbf{церковнославянский}.

Единственная сфера, в которой использовался исключительно древнерусский язык — юридическая.

\hfill

Наш \textbf{современный язык — результат слияния двух языков}, \textbf{древнерусского} и пришедшего с христианством, условно, \textbf{старославянского}.

В восемнадцатом веке начался процесс сознательного слияния двух языков, гармоничного синтеза.

Создание русской прозы — \textbf{Александр Сергеевич Пушкин} со своими друзьями. Середина девятнадцатого века — пик русской романистики.

\hfill

Во времена \textbf{Петра I} в русский язык вошло огромное количество новых слов посредством заимствования, \textbf{создание первой русской газеты}, \textbf{гражданский шрифт}.

\textbf{10 октября 1918 года} — реформа русской орфографии — готовилась с 1904 года — \textbf{устранение дублирующихся букв}

\subsection{Современное состояние русского языка}

Как происходило много \textbf{заимствований} раньше, так и сейчас мы заимствуем много всего из других языков.

Что менее заметно, но более важно, появляются новые конструкции и изменяются старые конструкции — \textbf{изменение грамматики}.

Изменился \textbf{темп речи}, он ускорился на треть за последние 40–60 лет. Изменения в \textbf{интонации}.

\textbf{Растабуированность нецензурной лексики}

\end{document}
